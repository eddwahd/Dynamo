\documentclass[aps, pra, a4paper]{revtex4}


\usepackage[utf8]{inputenc}
\usepackage[T1]{fontenc}
\usepackage[australian]{babel}
\usepackage{graphicx, hyperref, amsmath, amssymb, verbatim}

\newcommand{\I}{I}
\newcommand{\be}{\begin{equation}}
\newcommand{\ee}{\end{equation}}
\newcommand{\eq}{\Leftrightarrow}

\newcommand{\ket}[1]{\left| #1 \right \rangle}
\newcommand{\bra}[1]{\left \langle #1 \right|}
\newcommand{\braket}[2]{\left \langle #1 | #2 \right \rangle}
\newcommand{\ketbra}[2]{\left| #1 \right \rangle \left \langle #2 \right|}


\DeclareMathOperator{\tr}{tr}
\DeclareMathOperator{\re}{Re}

\newcommand{\dd}[2]{\frac{\mathrm{d} #1}{\mathrm{d} #2}}

\begin{document}


\section{Basics}

\subsection{System}

Governing equation:
\be
\dot{X} = -\underbrace{(A(t) +\sum_k u_k(t) B_k)}_{G(t)} X(t) = -G(t) X(t),
\ee
where $u_k$ are the (scalar) control fields, $B_k$ the corresponding generators,
$A$ the drift generator, $X(t)$ the ``system'' (vector or operator), and $G(t)$ the total generator.

Assume that the controls are
piecewise constant in time: $u_{c,j}$, with $n$~time slices in total,
and the duration of the $k$th slice is~$\tau_k$, such that
\be
t_k = t_0 + \sum_{j=1}^{k} \tau_j.
\ee
For each time slice we now get the propagator
\be
P_k := \exp(-G_k \tau_k),
\ee
with
\be
X_k := X(t_k) = \prod_{j=1}^{k} P_j X(t_0).
\ee
NOTE that this propagator always acts on the system by multiplication from the left. In Task 4 we make an exception on this rule.


System and adjoint system propagators:
\begin{align}
U_k &:= P_k \cdots P_1,\\      % U_k X_0 = X_k
\Lambda_k &:= P_n \cdots P_{k+1}.
\end{align}
NOTE: currently the code uses a different definition:
$U_{k+1} = P_k \cdots P_1 X_0 = X_k$ and
$\Lambda_k = X_f^\dagger P_n \cdots P_k$ (MATLAB indexing limitations...).

\begin{table}[h]
\[
\begin{array}{c}
\begin{array}{@{t_0}p{1.9em}@{t_1}p{1.8em}@{t_2}p{1.4em}@{t_{k-1}}p{1.1em}@{t_{k}}p{1.1em}@{t_{k+1}}p{0.8em}@{t_{n-2}}p{0.8em}@{t_{n-1}}p{1.3em}@{t_n}}
& & & & & & &
\end{array}\\
\begin{array}{|p{2em}|p{2em}|p{2em}|p{2em}|p{2em}|p{2em}|p{2em}|p{2em}|}
 $\tau_1$ & $\tau_2$ & & $\tau_k$ & $\tau_{k+1}$ & & $\tau_{n-1}$ & $\tau_n$ \\
% $u_{c,1}$ & $u_{c,2}$ & & $u_{c,k}$ & $u_{c,k+1}$ & & $u_{c,n-1}$ & $u_{c,n}$ \\
 $P_1$ & $P_2$ & $\cdots$ & $P_k$ & $P_{k+1}$ & $\cdots$ & $P_{n-1}$ & $P_n$ \\
\cline{1-4}
& & & $U_{k}$ & $\Lambda_{k}$ & & & \\
\cline{5-8}
\end{array}
\end{array}
\]
\caption{Time slices and operators related to them.
$t_k = t_0 + \sum_{j=1}^{k} \tau_j$.
The total forward and backward
propagators to the point $t_k$ are defined as
$U_k = P_k \cdots P_1$ and
$\Lambda_k = P_{n} \cdots P_{k+1}$.}
\end{table}



\subsection{Optimization goal}

We usually wish to minimize an operator/vector distance, measured using the Frobenius norm:
\be
d(A, B)^2 = |A-B|^2 = \tr((A-B)^\dagger (A-B))
= |A|^2 +|B|^2 -\tr(A^\dagger B) -\tr(B^\dagger A)
= |A|^2 +|B|^2 -2 \re \tr(A^\dagger B)
\ee

Normalize with $|A|^2$ (assumed fixed, known):
\be
\frac{d^2(A, B)}{|A|^2}
= 1 +\frac{|B|^2}{|A|^2} -\frac{2}{|A|^2} \re \tr(A^\dagger B)
\ee

Define normalized fidelity\footnote{There is another widely used quantity called fidelity in quantum information science which is different from the present one.}
(real function):
\be
f(A, B) := \frac{1}{|A|^2} \re \tr(A^\dagger B).
\ee
Clearly $f(A, A) = 1$. Now
\be
\frac{d^2(A, B)}{|A|^2} = 1 +\frac{|B|^2}{|A|^2} -2 f(A, B)
\ee
and
\be
d^2(A, B) \ge 0
\qquad \eq \qquad
f(A, B) \le \frac{1}{2} \left(1 +\frac{|B|^2}{|A|^2} \right).
\ee



(X1): Is minimum distance equivalent to maximum fidelity?
If this holds, we may optimize fidelity instead of distance.
If $|A|^2 = C_A$ and $|B|^2 = C_B$ are constant, we have
\be
d(A, B)^2 / C_A = 1 +C_B/C_A - 2 f(A, B)
\ee
$\implies$ (X1) holds.
For a unitary operator~$U$ this is always the case:
\be
|U|^2 = \tr(U^\dagger U) = \tr(\I) = N.
\ee


\subsection{Auxiliary function g}

Define
\be
g(A, B) := \frac{1}{|A|^2} \tr(A^\dagger B).
\ee

We will encounter two common quality functions to be maximized:
\begin{align}
f(A,B) &:= \re g(A,B), \: \text{and}\\
Q(A,B) &:= |g(A,B)|.
\end{align}

Limits:
\begin{align}
|f(A, B)| &\le \frac{1}{2} \left(1 +\frac{|B|^2}{|A|^2} \right),\\
0 &\le Q^2(A,B) \le \frac{1}{2} \left(1 +\frac{|B|^4}{|A|^4} \right).
\end{align}
(The upper limit for $Q$ is obtained using the lifting trick in Task 1.)
Gradients:
\begin{align}
\dd{f(A,B)}{u}
&= \re \left( \dd{g}{u} \right) \\
\dd{Q(A,B)}{u}
%= \dd{|g(X_f, X_n)|}{u_k(t_j)}
&= \re \left(\frac{g^*}{|g|} \dd{g}{u} \right).
\end{align}



\section{Optimization tasks}

\subsection{Task 1}

Unitary operator $X$ up to phase, unitary propagation.

We can get rid of phase by lifting the problem into Liouville space:
$\hat{U} := U^* \otimes U$.
We then have for any~$A$, $B$
\be
\tr(\hat{A}^\dagger \hat{B})
= \tr((A^* \otimes A)^\dagger (B^* \otimes B))
= \tr((A^T B^*) \otimes (A^\dagger B))
= \tr((A^\dagger B)^* \otimes (A^\dagger B))
= |\tr(A^\dagger B)|^2,
\ee
and thus
\be
|\hat{A}|^2 = \tr(\hat{A}^\dagger \hat{A})
= |\tr(A^\dagger A)|^2
= |A|^4.
\ee
Hence for a unitary~$U$ we have
$|\hat{U}|^2 = |U|^4 = N^2$.

Minimize operator distance $d(\hat{X}_{f}, \hat{X}_n)$:
\be
\frac{d^2(\hat{X}_f, \hat{X}_n)}{|\hat{X}_f|^2} 
= 1 +\frac{|\hat{X}_n|^2}{|\hat{X}_f|^2} -2 f(\hat{X}_f, \hat{X}_n)
= 2 -2 f(\hat{X}_f, \hat{X}_n)
%= 2 -\frac{2}{N^2} (\re) |\tr(A^\dagger B)|^2
\ee
(X1) holds, and
the problem actually simplifies back into Hilbert space, but with an extra abs value squared:
\be
f(\hat{X}_f, \hat{X}_n)
= \frac{1}{N^2} (\re) |\tr(X_f^\dagger X_n)|^2
= Q^2(X_f, X_n).
\ee
Now, maximize $0 \le Q(X_f, X_n) \le 1$.


\subsection{Task 2}

Unitary operator $X$, phase matters, unitary propagation.
Minimize operator distance $d(X_f, X_n)$.

For a unitary operator~$U$,
$|U|^2 = N$, hence (X1) holds, and we can equivalently maximize the fidelity
$-1 \le f(X_f, X_n) \le 1$.


\subsection{Task 3}

Pure normalized state vector X, unitary propagation.
$|A|^2 = \braket{A}{A} = 1$.
Maximize state overlap
\be
|\braket{X_f}{X_n}|
= |(\tr)(X_f^\dagger X_n)|
= Q(X_f, X_n),
\ee
$0 \le Q(X_f, X_n) \le 1$.


If global phase matters, minimize $d(X_f, X_n)$, and since (X1) holds,
maximize the fidelity:
$-1 \le f(X_f, X_n) \le 1$.



\subsection{Task 4}

General state operator X, unitary propagation.
Minimize state operator distance $d(X_f, X_n)$.

\be
|\rho|^2
= \tr(\rho^\dagger \rho)
= \tr(\rho^2)
= P(\rho)
\ee
Unitary propagation conserves purity, hence (X1) holds.
Maximize fidelity:
\be
0 \le f(X_f, X_n) \le \frac{1}{2} \left(1 +\frac{P(X_0)}{P(X_f)}\right).
\ee
The lower limit on the fidelity results from state operators being positive.

%\be
%Q(X_f, X_n)
%= \frac{1}{P(X_f)} (\re) \tr(X_f X_n)
%= 1/P_f  (\re) tr(X_f  P_n ... P_1 X_0 P_1^\dagger ... P_n^\dagger)
%\ee

In this task we exceptionally avoid Liouville space using a more complicated propagator:
\be
X_i := \left(\prod_{j=1}^{i} P_j\right) X_0 \left(\prod_{j=1}^{i} P_j\right)^\dagger
\ee

\begin{align}
\dd{f(X_f, X_n)}{u(t_j)}
&= \re \left(\dd{g}{u(t_j)} \right)
= \frac{1}{P(X_f)} (\re) \tr \left(X_f \dd{X_n}{u(t_j)}\right)\\
&= \frac{1}{P(X_f)} \left(\tr \left(X_f P_n \cdots \dd{P_j}{u(t_j)} \cdots P_1 X_0 P^\dagger_1 \cdots P^\dagger_n\right)
+\tr\left(X_f P_n \cdots P_1 X_0 P^\dagger_1 \cdots \dd{P^\dagger(t_j)}{u(t_j)} \cdots P^\dagger_n\right)\right)\\
&= \frac{2}{P(X_f)} \re \tr\left(X_f P_n \cdots \dd{P_j}{u(t_j)} \cdots P_1 X_0 P^\dagger_1 \cdots P^\dagger_n\right).
\end{align}
NOTE: last line not in paper!


NOTE: if both states are pure, $X = \ketbra{X}{X}$, we obtain
\be
f(X_f, X_n)
= (\re)\left|(\tr) \bra{X_f}  P_n \cdots P_1 \ket{X_0} \right|^2
= Q^2(\ket{X_f}, \ket{X_n}).
\ee
(same as Task 3, irrelevant global phase, but with square.)


\subsection{Task 5}

General quantum map operator $X$, Markovian propagation
in Liouville space.
Minimize operator distance $d(X_f, X_n)$.

A unitary target map,
$X_f = \hat{U}$, for example, gives $|X_f|^2 = N^2$.

However, the norm of the propagated operator is not necessarily constant:
\be
|X_n|^2 = \tr\left(X_0^\dagger \left(\prod_{j=1}^{n} P_j\right)^\dagger \left(\prod_{j=1}^{n} P_j\right) X_0\right).
\ee
If $G_j$ is normal $\eq \quad [G_j, G^\dagger_j] = 0$, we have
\be
P_j^\dagger P_j
= \exp(-\tau_j G^\dagger_j) \exp(-\tau_j G_j)
= \exp(-\tau_j (G_j^\dagger + G_j)).
\ee
If all the generators $G_j$ are antihermitian, this reduces to $\I$, and thus
$|X_n|^2 = \tr(X_0^\dagger X_0) = |X_0|^2$.

If this is not the case, (X1) is not satisfied and fidelity does not uniquely define the distance.
If it does,
maximize it:
\be
|f(X_f, X_n)| \le \frac{1}{2} \left(1 +\frac{|X_n|^2}{|X_f|^2} \right).
\ee

Otherwise, directly minimize
$0 \le d^2(A, B)/|A|^2$
or equivalently maximize
\be
W(A,B) := 2 f(A, B) -\frac{|B|^2}{|A|^2} = 1-d^2(A, B)/|A|^2 \le 1.
\ee
The fidelity derivative is computed as usual, the norm derivative gives
\be
\dd{|B|^2}{u(t_j)}
= \dd{}{u(t_j)} \tr(B^\dagger B)
= \tr\left(B^\dagger \dd{B}{u(t_j)}\right)
+\tr\left(\dd{B^\dagger}{u(t_j)} B \right)
= 2 \re \tr\left(B^\dagger \dd{B}{u(t_j)}\right).
\ee
and thus
\be
\dd{W(X_f, X_n)}{u(t_j)}
= \frac{2}{|X_f|^2} \re \tr\left((X_f-X_n)^\dagger \dd{X_n}{u(t_j)}\right).
\ee


\subsection{Task 6}
General vectorized state operator $X = \text{vec}(\rho)$, Markovian propagation
in Liouville space.

Minimize operator distance
\be
d(X_f, X_n) = d(\text{vec}(\rho_f), \text{vec}(\rho_n)) = d(\rho_f, \rho_n).
\ee

$|\rho|^2 = P(\rho)$, but general Markovian propagation does not preserve purity!  
So, like before, we need to use a bit more complicated quality func...

\end{document}
