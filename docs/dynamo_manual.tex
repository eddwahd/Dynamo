\documentclass[aps, pra, a4paper, longbibliography]{revtex4}
% [VB] needs to be changed to revtex4-1 but my current latex setup is
% ancient so we'll need to use this for a while

\usepackage[utf8]{inputenc}
\usepackage[T1]{fontenc}
\usepackage[australian]{babel}
\usepackage{graphicx, hyperref, amsmath, amssymb, verbatim}

\newcommand{\I}{I}
\newcommand{\be}{\begin{equation}}
\newcommand{\ee}{\end{equation}}
\newcommand{\eq}{\Leftrightarrow}

\newcommand{\ket}[1]{\left| #1 \right \rangle}
\newcommand{\bra}[1]{\left \langle #1 \right|}
\newcommand{\braket}[2]{\left \langle #1 | #2 \right \rangle}
\newcommand{\ketbra}[2]{\left| #1 \right \rangle \left \langle #2 \right|}
\newcommand{\comm}[2]{\left[ #1, #2 \right]}

\newcommand{\hilb}[1]{\mathcal{#1}}

\DeclareMathOperator{\tr}{tr}
\DeclareMathOperator{\re}{Re}
\DeclareMathOperator{\cvec}{vec}

\newcommand{\dd}[2]{\frac{\partial #1}{\partial #2}}

\begin{document}

\tableofcontents

\section{Basics}

Dynamo~\cite{machnes_2011},


\subsection{System}

Dynamo can handle several types of linear systems, all of which have a
master equation of the following type:
\be
\label{eq:master}
\dot{X} = -\underbrace{(A(t) +\sum_k u_k(t) B_k)}_{G(t)} X(t) = -G(t) X(t).
\ee
$u_k$ are the (scalar) control fields, $B_k$ the corresponding generators,
$A$ the drift generator, $X$ the ``system'' (vector or operator), and $G$ the total generator.

Assume that the controls are
piecewise constant in time: $u_{c,j}$, with $n$~time slices in total,
and that the duration of the $k$th slice is~$\tau_k$, giving
\be
t_k = t_0 + \sum_{j=1}^{k} \tau_j.
\ee
For each time slice we now obtain the propagator
\be
P_k := \exp(-G_k \tau_k),
\ee
with
\be
X_k := X(t_k) = \prod_{j=1}^{k} P_j X(t_0).
\ee
This propagator, like the generators, always acts on the system by multiplication
from the left.

Derivatives:
\begin{align}
\dd{P_k}{\tau_k}  &= -G_k P_k = -P_k G_k,\\
\dd{P_k}{u_{c,k}}
&=
\sum_{j=0}^{\infty} \frac{(-\tau_k)^j}{j!}
\sum_{q=0}^{j-1}
G_k^{q} \dd{G_k}{u_{c,k}} G_k^{j-q-1}
=
\sum_{j=0}^{\infty} \frac{(-\tau_k)^j}{j!}
\sum_{q=0}^{j-1}
G_k^{q} B_c G_k^{j-q-1}
\approx -\tau_k B_c P_k.
\end{align}
The approximation is exact if $\comm{B_c}{G_k} = 0$.


System and adjoint system propagators:
\begin{align}
U_k &:= P_k \cdots P_1,\\      % U_k X_0 = X_k
\Lambda_k &:= P_n \cdots P_{k+1}.
\end{align}
NOTE: currently the code uses a different definition:
$U_{k+1} = P_k \cdots P_1 X_0 = X_k$ and
$\Lambda_k = X_f^\dagger P_n \cdots P_k$ (MATLAB indexing limitations...).

\begin{table}[h]
\[
\begin{array}{c}
\begin{array}{@{t_0}p{1.9em}@{t_1}p{1.8em}@{t_2}p{1.4em}@{t_{k-1}}p{1.1em}@{t_{k}}p{1.1em}@{t_{k+1}}p{0.8em}@{t_{n-2}}p{0.8em}@{t_{n-1}}p{1.3em}@{t_n}}
& & & & & & &
\end{array}\\
\begin{array}{|p{2em}|p{2em}|p{2em}|p{2em}|p{2em}|p{2em}|p{2em}|p{2em}|}
 $\tau_1$ & $\tau_2$ & & $\tau_k$ & $\tau_{k+1}$ & & $\tau_{n-1}$ & $\tau_n$ \\
% $u_{c,1}$ & $u_{c,2}$ & & $u_{c,k}$ & $u_{c,k+1}$ & & $u_{c,n-1}$ & $u_{c,n}$ \\
 $P_1$ & $P_2$ & $\cdots$ & $P_k$ & $P_{k+1}$ & $\cdots$ & $P_{n-1}$ & $P_n$ \\
\cline{1-4}
& & & $U_{k}$ & $\Lambda_{k}$ & & & \\
\cline{5-8}
\end{array}
\end{array}
\]
\caption{Time slices and operators related to them.
$t_k = t_0 + \sum_{j=1}^{k} \tau_j$.
The total forward and backward
propagators to the point $t_k$ are defined as
$U_k = P_k \cdots P_1$ and
$\Lambda_k = P_{n} \cdots P_{k+1}$.}
\end{table}




\subsection{Hilbert space vs. Liouville space}

Equation~\eqref{eq:master} can refer to either Hilbert space or
Liouville space entities. Liouville space is used
for open systems, and the $\cvec$-representation of state operators in
closed systems.

The Liouville space equivalent for a unitary Hilbert space
propagator~$U$ is $\hat{U} := U^* \otimes U$.
We then have for any~$A$, $B$
\be
\label{eq:hat-product}
\tr(\hat{A}^\dagger \hat{B})
= \tr((A^* \otimes A)^\dagger (B^* \otimes B))
= \tr((A^T B^*) \otimes (A^\dagger B))
= \tr((A^\dagger B)^* \otimes (A^\dagger B))
= |\tr(A^\dagger B)|^2.
\ee



\subsection{Optimization goal}

We usually wish to minimize an operator/vector distance, measured using the Frobenius norm:
\be
d^2(A, B) = |A-B|^2 = \tr((A-B)^\dagger (A-B))
%= |A|^2 +|B|^2 -\tr(A^\dagger B) -\tr(B^\dagger A)
= |A|^2 +|B|^2 -2 \re \tr(A^\dagger B).
\ee
Normalizing this expression with the target operator norm $|A|^2$ (assumed fixed, known),
and introducing the normalized \emph{fidelity}\footnote{
There is another widely used quantity called fidelity in quantum information science which is different from the present one.},
\be
f(A, B) := \frac{1}{|A|^2} \re \tr(A^\dagger B),
\ee
we obtain the normalized distance measure
\be
\label{eq:df}
D(A,B)
:= \frac{d^2(A, B)}{|A|^2}
= 1 +\frac{|B|^2}{|A|^2} -2 f(A, B).
\ee
Clearly $f(A, A) = 1$, and
the triangle inequality $|A-B| \le |A|+|B|$ yields
\be
\left(1 -\frac{|B|}{|A|} \right)^2 \le D(A, B) \le \left(1 +\frac{|B|}{|A|} \right)^2
\qquad \text{and} \qquad
|f(A, B)|
\le \frac{|B|}{|A|}
%\le \frac{1}{2} \left(1 +\frac{|B|^2}{|A|^2} \right)
\ee
since $f(A, -B) = -f(A, B)$.


(X1): Is minimum distance equivalent to maximum fidelity?
If this holds, we may optimize fidelity instead of distance.
If $|A|^2$ and $|B|^2$ are constant
from Eq.~\eqref{eq:df} we can see that (X1) clearly holds.



\subsection{Auxiliary function g}

Define
\be
g(A, B) := \frac{1}{|A|^2} \tr(A^\dagger B).
\ee
We will encounter two common quality functions to be maximized:
\begin{align}
f(A,B) &:= \re g(A,B), \: \text{and}\\
Q(A,B) &:= |g(A,B)|.
\end{align}

%Limits:
%\begin{align}
%0 &\le Q^2(A,B) \le \frac{1}{2} \left(1 +\frac{|B|^4}{|A|^4} \right).
%\end{align}
%(The upper limit for $Q$ is obtained using the lifting trick in Task 1.)
Gradients:
\begin{align}
\dd{f(A,B)}{u}
&= \re \left( \dd{g}{u} \right) \\
\dd{Q(A,B)}{u}
%= \dd{|g(X_f, X_n)|}{u_k(t_j)}
&= \re \left(\frac{g^*}{|g|} \dd{g}{u} \right).
\end{align}





\section{Optimization tasks}

In what follows, all quantum states are assumed to be normalized to unity.


\subsection{Closed system S}

In a closed system, the propagators~$P_k$ are always unitary.

\subsubsection{Mixed state transfer $\rho_0 \to \rho_f$ (task 4)}
\label{sec:closed-mixed}

To conform to the form of Eq.~\eqref{eq:master},
the state operators are treated as vectors in Liouville space, $X~=~\cvec(\rho)$.
\footnote{It will of course turn out that this is equivalent to treating them as operators in Hilbert
space, $X~=~\rho$ as far as the results are concerned.}
The goal here is to minimize state operator distance
$D(\cvec(\rho_f), \cvec(\rho_n)) = D(\rho_f, \rho_n)$.
The norm squared is equivalent to the purity of the state:
\be
|\cvec(\rho)|^2
= \tr(\rho^\dagger \rho)
= \tr(\rho^2)
= P(\rho).
\ee
Unitary propagation conserves purity, hence (X1) holds, and we can
simply maximize the fidelity $f(X_f, X_n)$:
\be
f(X_f, X_n)
%= \frac{1}{P(\rho_f)} (\re) \tr(X_f^\dagger X_n)
= \frac{1}{P(\rho_f)} (\re) (\tr) \left( X_f^\dagger  P_n \cdots P_1 X_0 \right).
\ee
Furthermore, the fidelity is strictly nonnegative since the
state operators are positive:
\be
0 \le f(X_f, X_n) \le \sqrt{\frac{P(\rho_0)}{P(\rho_f)}}.
\ee
If either $\rho_f$ or $\rho_0$ is pure,
$\rho = \ketbra{\psi}{\psi}$,
we have $|\rho|^2 = \braket{\psi}{\psi}^2 = 1$, and
the diagram simplifies by splitting up.





TODO
Alternatively, we can choose $X = \rho$ at the expense of a slightly
more complicated expression for the fidelity:
\begin{comment}
\be
f(X_f, X_n)
= \frac{1}{P(X_f)} (\re) \tr(X_f^\dagger X_n)
= \frac{1}{P(X_f)} (\re) \tr(X_f^\dagger  P_n ... P_1 X_0 P_1^\dagger ... P_n^\dagger)
\ee

\be
X_i := \left(\prod_{j=1}^{i} P_j\right) X_0 \left(\prod_{j=1}^{i} P_j\right)^\dagger
\ee

\begin{align}
\dd{f(X_f, X_n)}{u(t_j)}
&= \re \left(\dd{g}{u(t_j)} \right)
= \frac{1}{P(X_f)} (\re) \tr \left(X_f \dd{X_n}{u(t_j)}\right)\\
&= \frac{1}{P(X_f)} \left(\tr \left(X_f P_n \cdots \dd{P_j}{u(t_j)} \cdots P_1 X_0 P^\dagger_1 \cdots P^\dagger_n\right)
+\tr\left(X_f P_n \cdots P_1 X_0 P^\dagger_1 \cdots \dd{P^\dagger(t_j)}{u(t_j)} \cdots P^\dagger_n\right)\right)\\
&= \frac{2}{P(X_f)} \re \tr\left(X_f P_n \cdots \dd{P_j}{u(t_j)} \cdots P_1 X_0 P^\dagger_1 \cdots P^\dagger_n\right).
\end{align}
NOTE: last line not in paper!
\end{comment}



\subsubsection{Pure state transfer $\ket{\psi_0} \to \ket{\psi_f}$ (task 3)}
\label{sec:closed-pure}

Using the results from the previous section with both states pure,
$\rho = \ketbra{\psi}{\psi}$, with
$X = \cvec(\rho) = \ket{\psi^*} \otimes \ket{\psi}$,
the fidelity diagram breaks into two pieces and
we obtain
\be
f(X_f, X_n)
= (\re) \left|(\tr) \bra{\psi_f}  P_n \cdots P_1 \ket{\psi_0} \right|^2.
\ee
with $0 \le f(X_f, X_n) \le 1$.
Thus the problem simplifies back into Hilbert space
(albeit with an extra absolute value squared in the expression for the fidelity), and we may equivalently
choose $X = \ket{\psi}$ and maximize $Q(X_f, X_n)$.


%Maximize state overlap
% TODO redefine X, in which space does P operate?


If global phase matters (NOTE: this is unphysical), we may define $X = \ket{\psi}$ and minimize
$D(X_f, X_n)$ directly, and since (X1) holds,
equivalently maximize the fidelity
\be
f(X_f, X_n)
= \re (\tr) \left(X_f^\dagger P_n \cdots P_1 X_0 \right),
\ee
obeying
$|f(X_f, X_n)| \le 1$.



\subsubsection{Unitary propagator (tasks 1 and 2)}

In this task we wish to generate a unitary gate~$V_f$ up to global
phase, starting from the identity~$V_0 = \I$.
We can get rid of phase by explicitly lifting the problem into Liouville space,
$X = \hat{V} = V^* \otimes V$,
and then minimize the operator distance~$D(X_f, X_n)$.

Using Eq.~\eqref{eq:hat-product}, the norm squared is 
\be
|\hat{V}|^2
= \tr(\hat{V}^\dagger \hat{V})
= |\tr(V^\dagger V)|^2
= |\tr(\I)|^2
= N^2,
\ee
where $N = \dim \hilb{H}$.
%\be
%|U|^2 = \tr(U^\dagger U) = \tr(\I) = N.
%\ee
This is constant, so (X1) holds and we may maximize the fidelity instead:
\be
f(X_f, X_n)
= \frac{1}{N^2} \re \tr \left(\hat{V_f}^\dagger \hat{V_n} \right)
%= \frac{1}{N^2} (\re) \left| \tr \left(V_f^\dagger V_n \right) \right|^2
= \frac{1}{N^2} (\re) \left| \tr \left(V_f^\dagger P_n \cdots P_1 V_0 \right) \right|^2.
\ee
Much like in
Sec.~\ref{sec:closed-pure},
the problem simplifies back into Hilbert space, and we may equivalently
choose $X = V$ and
maximize $0 \le Q(X_f, X_n) \le 1$.


If global phase matters (NOTE: unphysical), we may choose~$X = V$ and directly minimize
the operator distance~$D(X_f, X_n)$.
For a unitary operator~$V$ we have $|V|^2 = N$, hence (X1) holds, and
we can equivalently maximize the fidelity
\be
f(X_f, X_n)
= \frac{1}{N} \re \tr \left(X_f^\dagger P_n \cdots P_1 X_0 \right)
\ee
with $|f(X_f, X_n)| \le 1$.



\subsection{Open system with bath S+B}

In this case we have a system~$S$ coupled to a Markovian bath~$B$.
The propagators~$P_k$ are now expressible using Lindblad operators.


\subsubsection{Mixed state transfer $\rho_0 \to \rho_f$ (task 6)}

Like in Sec.~\ref{sec:closed-mixed} we choose $X~=~\cvec(\rho)$ and
minimize $D(\cvec(\rho_f), \cvec(\rho_n)) = D(\rho_f, \rho_n)$.
Likewise, we have
$|\rho|^2 = P(\rho)$, but general Markovian propagation does not
preserve purity, so (X1) does not hold.

So, like before, we need to use a bit more complicated quality func...



\subsubsection{General quantum map (task 5)}

General quantum map operator $X = F$,
minimize operator distance $D(X_f, X_n)$.

A unitary target map,
$X_f = \hat{V}$, for example, gives $|X_f|^2 = N^2$.
However, the norm of the propagated operator~$X_n$ is not necessarily constant:
\be
|X_n|^2 = \tr\left(X_0^\dagger \left(\prod_{j=1}^{n} P_j\right)^\dagger \left(\prod_{j=1}^{n} P_j\right) X_0\right).
\ee
If $G_j$ is normal $\eq \quad [G_j, G^\dagger_j] = 0$, we have
\be
P_j^\dagger P_j
= \exp(-\tau_j G^\dagger_j) \exp(-\tau_j G_j)
= \exp(-\tau_j (G_j^\dagger + G_j)).
\ee
If all the generators $G_j$ are antihermitian, this reduces to $\I$, and thus
$|X_n|^2 = \tr(X_0^\dagger X_0) = |X_0|^2$.

If this is not the case, (X1) is not satisfied and fidelity does not uniquely define the distance.
If it does, maximize it:
\be
|f(X_f, X_n)| \le \frac{|X_0|}{|X_f|}.
\ee
Otherwise, directly minimize~$D(X_f, X_n)$:
\be
D(X_f, X_n) = 1 +\frac{|X_n|^2}{|X_f|^2} -2 f(X_f, X_n).
\ee
The fidelity derivative is computed as usual, the norm derivative gives
\be
\dd{|B|^2}{u(t_j)}
= \dd{}{u(t_j)} \tr(B^\dagger B)
= \tr\left(B^\dagger \dd{B}{u(t_j)}\right)
+\tr\left(\dd{B^\dagger}{u(t_j)} B \right)
= 2 \re \tr\left(B^\dagger \dd{B}{u(t_j)}\right).
\ee
and thus
\be
\dd{D(X_f, X_n)}{u(t_j)}
= \frac{2}{|X_f|^2} \re \tr\left((X_n-X_f)^\dagger \dd{X_n}{u(t_j)}\right).
\ee


\subsection{Closed system and environment S+E}

In this case we have a system~$S$ coherently coupled to an
environment~$E$, but are only interested in the state of the system.
The $S$+$E$ propagators~$P_k$ are unitary.

partial distance measures~\cite{kosut_2006}.

\subsection{Open system and environment coupled to a bath S+E+B}

TODO



\bibliography{dynamo}
\end{document}
